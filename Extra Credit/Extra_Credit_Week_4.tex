\documentclass[12pt]{article}
\textwidth = 6.5 in
\textheight = 8 in
\oddsidemargin = 0.0 in
\evensidemargin = 0.0 in
\topmargin = 0.2 in
\headheight = 0.0 in
\headsep = 1.1 in
\parskip = 0.2in
\parindent = 0.0in

\usepackage{amsmath,amssymb,amsthm,enumerate,fancyhdr,mathtools,graphics,setspace}
\usepackage{listings}
\usepackage{color}

\definecolor{dkgreen}{rgb}{0,0.6,0}
\definecolor{gray}{rgb}{0.5,0.5,0.5}
\definecolor{mauve}{rgb}{0.58,0,0.82}

\lstset{frame=tb,
  aboveskip=3mm,
  belowskip=3mm,
  showstringspaces=false,
  columns=flexible,
  basicstyle={\small\ttfamily},
  numbers=none,
  language=Java,
  numberstyle=\tiny\color{gray},
  keywordstyle=\color{blue},
  commentstyle=\color{dkgreen},
  stringstyle=\color{mauve},
  breaklines=true,
  breakatwhitespace=true,
  tabsize=3
}
\pagestyle{fancy}
\rhead{Zhihao Xue\\CM 3357\\Extra Credit Week 7\\CSSE490  Dr. Mohan\\Thursday,Oct 22, 2015}

%=======================================================================
\begin{document}
	\begin{itemize}
		\item How to specify and use a hive user defined aggregate function (UDAF) for hive with a code sample. \\
		An aggregate function is more difficult to write than  a regular UDF. Values are aggregated in chunks, so the implementation has to be capable of combining partial aggregation into a final result.           
		\begin{verbatim}
package com.hadoopbook.hive;

import org.apache.hadoop.hive.ql.exec.UDAF;
import org.apache.hadoop.hive.ql.exec.UDAFEvaluator;
import org.apache.hadoop.io.IntWritable;

public class Maximum extends UDAF {

  public static class MaximumIntUDAFEvaluator implements UDAFEvaluator {
    
    private IntWritable result;
    
    public void init() {
      System.err.printf("%s %s\n", hashCode(), "init");
      result = null;
    }

    public boolean iterate(IntWritable value) {
      System.err.printf("%s %s %s\n", hashCode(), "iterate", value);
      if (value == null) {
        return true;
      }
      if (result == null) {
        result = new IntWritable(value.get());
      } else {
        result.set(Math.max(result.get(), value.get()));
      }
      return true;
    }

    public IntWritable terminatePartial() {
      System.err.printf("%s %s\n", hashCode(), "terminatePartial");
      return result;
    }

    public boolean merge(IntWritable other) {
      System.err.printf("%s %s %s\n", hashCode(), "merge", other);
      return iterate(other);
    }

    public IntWritable terminate() {
      System.err.printf("%s %s\n", hashCode(), "terminate");
      return result;
    }
  }
}
                                                                                                                                                                                                                                                                                                                                                                                                                                                             
		                                                                                                                                                                                                                                     
		\end{verbatim}
		
		\begin{lstlisting}[language=bash]
		hive> CREATE TEMPORARY FUNCTION maximum AS 'com.hadoopbook.hive.Maximun';
		hive> SELECT maximum(temperature) FROM records;
		\end{lstlisting}	
	\end{itemize}
	
\end{document}